\documentclass{article}

\title{Macroeconomic Models: \\ Weekly Update}
\author{Chris Comiskey, Open Data Group}
\date{\today}

\usepackage{natbib}
\bibliographystyle{unsrtnat}

\usepackage{fullpage}
\usepackage{ulem}

\usepackage{amsmath, amsthm, amssymb, amsfonts}
\usepackage{mathtools}
\usepackage{float}
\usepackage{bbm}

\usepackage{listings}


\begin{document}

\maketitle{}

\begin{itemize}
\item Pretty easy to see the {\bf strong} seasonality in the ACF of CPI's month-to-month changes.
  \begin{figure}[H]
  \centering
  \includegraphics[scale=.15]{/Users/cwcomiskey/Desktop/ODG/Macro-models/plots/CPImtmACF.jpg}
  \end{figure}
\item Linear regression model residuals:
      \begin{figure}[H]
      \centering
      \includegraphics[scale=.12]{/Users/cwcomiskey/Desktop/ODG/Macro-models/plots/ModelResiduals.jpg}
      \end{figure}
\item Model residuals: empirical autocorrelation function (ACF) plot.
      \begin{figure}[H]
      \centering
      \includegraphics[scale=.12]{/Users/cwcomiskey/Desktop/ODG/Macro-models/plots/ModResidACF.jpg}
      \end{figure}
\item Model residuals: empirical partial autocorrelation function (ACF) plot.
      \begin{figure}[H]
      \centering
      \includegraphics[scale=.12]{/Users/cwcomiskey/Desktop/ODG/Macro-models/plots/ModResidPACF.jpg}
      \end{figure}
\item \verb|auto.arima()| ranks a simple AR(1) as the top SARMIA model for the residuals. I modeled the linear regression residuals as AR(1), and the following plot shows the ACF for the (new) resulting residuals.
      \begin{figure}[H]
      \centering
      \includegraphics[scale=.15]{/Users/cwcomiskey/Desktop/ODG/Macro-models/plots/Mod+AR1Resid.jpg}
      \end{figure}
\item HOWEVER... dividing the data into training and testing subsets gives a different result; the training set residuals do not show enough autocorrelation to justify modeling them as AR(1).
\item Last month's CPI change is not a sifnificant predictor of this month's CPI change; see last row, starred for visibility:
\begin{verbatim}
lm(formula = CPI ~ ., data = select(reg_dat, -date))

Coefficients:
              Estimate Std. Error t value Pr(>|t|)    
(Intercept)  0.0405137  0.0435696   0.930 0.355550    
SETB         0.0423077  0.0009428  44.877  < 2e-16 
SEHB         0.0196479  0.0036115   5.440 6.96e-07
.
.
.
SEHJ         0.0258326  0.0180102   1.434 0.155806    
last_month   0.0026682  0.0196039   0.136 0.892119 (****)
---

Residual standard error: 0.09496 on 72 degrees of freedom
Multiple R-squared:  0.9866,	Adjusted R-squared:  0.9817 
F-statistic: 203.1 on 26 and 72 DF,  p-value: < 2.2e-16
\end{verbatim}

\item The same is true of the CPI change 12 months before.
\begin{verbatim}
lm(formula = CPI ~ ., data = select(reg_dat, -date))

Coefficients:
              Estimate Std. Error t value Pr(>|t|)    
(Intercept)  0.0450185  0.0424584   1.060 0.292554    
SETB         0.0423946  0.0009511  44.574  < 2e-16 
SEHB         0.0199526  0.0036733   5.432 7.20e-07
.
.
.
SEHJ         0.0257108  0.0178891   1.437 0.154984    
lag12_delta -0.0062173  0.0197272  -0.315 0.753550 (****)
---

Residual standard error: 0.09491 on 72 degrees of freedom
Multiple R-squared:  0.9866,	Adjusted R-squared:  0.9817 
F-statistic: 203.4 on 26 and 72 DF,  p-value: < 2.2e-16
\end{verbatim}
\end{itemize}
\section*{Relative Importance Weights}
\begin{itemize}
\item https://www.bls.gov/cpi/tables/relative-importance/home.htm
\end{itemize}
\end{document}